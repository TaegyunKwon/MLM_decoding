\documentclass{article}
\usepackage[utf8]{inputenc}

\title{Music Language Model Decoding for AMT}
\author{Adrien Ycart, Andrew McLeod}
\date{January 2018}

\usepackage{natbib}
\usepackage{graphicx}
\usepackage{a4wide}
\usepackage{hyperref}

\begin{document}

\maketitle

\section{Introduction}

The goal of this project is to investigate language-model decoding in the context of AMT.
The general idea is to use a symbolic model of music to assess the likelihood of candidate solutions, obtained via an acoustic model.

This process is different from \cite{ycart2018polyphonic}.
In that paper, an LSTM was trained to convert acoustic model output to binary piano-rolls.
The model was trained on pairs of outputs of a specific acoustic model (obtained from audio files) and MIDI transcriptions of the audio file.
It required aligned pairs of audio recordings and MIDI transcriptions.
It was also trained for one specific acoustic model.

In the current study, the language model is trained on symbolic data only.
It can also be used with any acoustic model (unless some specific fine-tuning is done on the language model, see further down)
It is similar to what it done in \cite{sigtia2016end}.
The main difference is that we will investigate the effect of various time-steps on performance:
 \cite{sigtia2016end} used a 10ms timestep, which we argue is not suited for music language modelling (see \cite{Ycart2017} and \cite{Korzeniowski2017}).
The language model will also have a different architecture: 
\cite{sigtia2016end} used the RNN-RBM \cite{Boulanger-Lewandowski2012}, while we will use a simple LSTM, similar to \cite{Ycart2017}.

A similar language-model decoding was also presented in \cite{app8030470} for AMT.
They used the RNN-RBM as language model, but using an "event" timestep (i.e. one timestep per new onset, without considering the duration of each note).
It also echoes \cite{Korzeniowski2018} in the context of chord recognition.
We also aim to evaluate the performance using this kind of timestep.

\section{Models}

\subsection{Acoustic model}

As acoustic model, we will use \cite{Kelz2016}.
It is a state-of-the-art model, only recently surpassed by \cite{Hawthorne2018}.
It also uses the same architecture as in \cite{sigtia2016end}, with only slight modifications.
It outputs pitch likelihoods as independent Bernouilli variables.

\subsection{Language model}

As language model, we will use a simple LSTM, as described in \cite{Ycart2017}.
The main argument for using it is to determine whether the interesting qualitative properties displayed by the 16th note model on symbolic prediction can bring some kind of improvement in the context of AMT.


\section{General workflow}
\label{sec:workflow}

Our system will operate on acoustic model outputs, in the form of a $88\times T$ matrix $M$,
with real values between 0 and 1.
It will output a matrix of same size, with binary values.

The system will find the best global solution by keeping a heap of candidate binary pianorolls
for timesteps $0,..,t$.
At each iteration, it will update this heap of candidates with possible continuations at timestep $t+1$.
At timestep $T-1$, it will select the candidate with highest likelihood as the global solution.

More specifically, at each timestep $t$, the general algorithm is the following:

\begin{enumerate}
\item Get $K$ binary continuation for timestep $t$ ($K$ is the branching factor)
\item For each candidate in the heap, compute the likelihood of the candidate concatenated with the continuation
\item Update the heap based with these continued candidates
\end{enumerate}

\section{Datasets}

The acoustic model is trained on the MAPS dataset \cite{emiya2010multipitch}.

The language model will be trained using different kinds of data.
Unquantised (i.e. performance) MIDI data will be obtained from the International Piano-e-Competition\footnote{\url{http://piano-e-competition.com/}},
and from the Piano-Midi.de dataset\footnote{\url{http://piano-midi.de/}}.
Quantised data will be taken from Boulanger-Lewandowski's datasets\footnote{\url{http://www-etud.iro.umontreal.ca/~boulanni/icml2012}},
from the Piano-Midi.de dataset (more recent version than the one used by Boulanger-Lewandowski),
as well as from MIDI data scraped from various websites.
The scraped data will be checked to make sure it corresponds to piano solo music, and that it is indeed quantised.
We will also make sure that it doesn't contain pieces that are in the MAPS test set.

For testing and for training in the proof-of-concept experiment (see \ref{sec:POC}), the A-MAPS \cite{ycart2018maps} rhythm annotations will be used.

\section{Experiments}

\subsection{Preliminary experiment}

We will run a quick experiment to show that a transduction model is fitted to one specific acoustic model, and cannot be used on a different one.
To do so, we will train the model from \cite{ycart2018polyphonic} using outputs from Kelz's model \cite{Kelz2016}.
We will then test it with another acoustic model, probably just a slightly modified version of  \cite{Kelz2016} (for instance with fewer layers).
We expect to see no improvement, maybe even a decrease in performance from using LSTM transduction decoding in the mismatched configuration, despite using only slightly different acoustic model.

\subsection{Proof-of-concept experiment}
\label{sec:POC}

We will compare 3 kinds of timesteps for the language model decoding:

\begin{enumerate}
\item 10ms timesteps, as in \cite{sigtia2016end}
\item 16th-note timesteps, as recommanded in \cite{Ycart2017}
\item event timesteps, as in \cite{app8030470}
\end{enumerate}

In this first experiment, we will use rhythm ground truth annotations from A-MAPS for 16th note timesteps, and ground-truth onset annotations for event timesteps.
The output of the acoustic model will be downsampled to the correct timestep using the \emph{step}
strategy described in \cite{ycart2018polyphonic}.

To compare all 3 timesteps, results will have to be converted back to 10ms timesteps.
For fare comparison, we will also evaluate the performance of 10ms timesteps and event timesteps using 16th-note quantisation as a post processing step.

\subsection{Real-life setting experiment}

We will again compare the same 3 kinds of timesteps, but without using ground truth annotations.
For 16th note timesteps, beat annotations will be obtained with a beat-tracking algorithm (probably taken from the madmom library \cite{madmom}).
For event timesteps, onset annotations will be obtained with an onset detection algorithm (probably also taken from madmom).

Discussion of the results will include an investigation on the correlation between success of the pre-processing method (beat tracking and onset detection) and the performance of the system, fo each test file.
One problem is that when using ground-truth annotations, we will use real 16th note timesteps, as written on the score, while in the real-life setting, we will use a fourth of a beat (which is not the same thing in the case of ternary meter).
We will also investigate how the performance of the system varies depending of the time signature of the test pieces (the time signatures are available in A-MAPS).


\section{Challenges}

For each of the steps outlined in section \ref{sec:workflow}, we outline some challenges and ideas to overcome them.

\subsection{Get continuations at timestep $t$}

At each timestep, we will have to get a set of binary continuations.
A naive strategy would be to sample binary vectors from the acoustic model independent pitch-wise distributions.

Given a vector of independent Bernouilli variables, \cite{Boulanger-Lewandowski2013} gives an algorithm to enumerate binary vectors in decreasing order of likelihood.
We will use this method.
We still have to define which likelihoods to take.

We identify 3 main strategies:

\begin{enumerate}
\item Sample from acoustic model distributions only
\item Sample from the product of the acoustic model and language model distributions
\item Sample from acoustic model on one side, and from the language model on the other, and use the union of these two sets of continuations
\end{enumerate}

In the first case, a note not detected by the acoustic model cannot be present in the output; the language model decoding can only eliminate false positives.
Also, the set of continuations will be the same for each candidate sequence.
In the two last cases, the set of continuation will depend on the language model, and thus on the candidate sequences, which might represent a potentially significant extra computational cost.

Experiments will include a comparison of these strategies.

\subsection{Compute the likelihood of the extended sequences}

The likelihood will be obtained as the product of acoustic model and language model likelihoods (potentially with a weighting factor between the two).

Problem: MLM trained on perfect sequences, tested on imperfect sequences
Solution: "teacher forcing"/"Scheduled Sampling" \cite{Bengio2015}
Replace some timesteps with solutions sampled from some probability distribution: either language model (but doesn't exactly correspond to test case), or from acoustic model (but then, MLM training is not independent from acoustic model)

\subsection{Update the heap}

Problem: potentially many near-duplicates
Solutions: ???



\bibliographystyle{plain}
\bibliography{references}
\end{document}
